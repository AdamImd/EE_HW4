\section{Text to image Generation}

\subsection{(A)No Prompt}
% (a) [2 pts] Run your pipeline using ωCFG = 0.0, η = 0.0 and 50 sampling steps without
% any positive or negative prompt to make sure everything works (this should give you
% a meaningful image generated unconditionally).

For the no-prompt generation, I used the same model as in part B, with the parameters $\omega CFG = 0.0$, num steps = 50, and $eta = 0.0$, with a fixed seed at 42. The resulting image is shown below.

\begin{figure}[h]
    \centering
    \includegraphics[width=0.4\textwidth]{../results/step3_results/part_a/part_a_unconditional.png}
    \caption{No Prompt Image Generation}
\end{figure}



\subsection{(B-C) 5X3 prompts- Manual evaluation}

% (b) [6 pts] Create 15 engaging prompts based on 5 unique and different topics (see some
% examples in (b) figure). For each topic, write three versions of the same idea:
% 1. Simple and short: a short, clear prompt (about one sentence).
% 2. Medium: a slightly longer prompt that adds some detail or context.
% 3. Long and complex: a detailed, imaginative version that expands on the idea
% with creative or descriptive elements.
% 2
% https://huggingface.co/docs/diffusers
% 6You will end up with 5 topics × 3 versions = 15 prompts in total. Do not rely on
% LLMs to generate you these prompts and use your creativity to generate interesting
% prompts that you want to see visualized as images. For each prompt, generate the
% corresponding SD output. Use ωCFG = 10.0 with 50 sampling steps and η = 0.0. Keep
% the negative prompt NULL. Finally, assign each generated image a similarity score (out
% of 10) based on how well it matches the intended prompt (e.g., 5.4/10.0). Share your
% thoughts. For instance, which version of a prompt typically gave the best similarity
% or image quality? Does prompt length affect fidelity to the intended concept or the
% final image quality?
I implemented a pipeline that will read from a set of prompts, and then use them to generate the images from the diffusers StableDiffusionPipeline module, with the $\omega CFG = 10$, num steps = 50, $eta = 0.0$, and a fixed seed at 42. The prompts for each image are included in the caption of the figures. 

\begin{itemize}
    \item Space
        \begin{itemize}
            \item Simple: Human: 5/10, CLIP: 28.49
            \item Medium: Human: 8/10, CLIP: 32.65
            \item Detailed: Human: 9/10, CLIP: 33.75
        \end{itemize}
    \item Ocean
        \begin{itemize}
            \item Simple: Human: 9/10, CLIP: 31.39
            \item Medium: Human: 6/10, CLIP: 32.93
            \item Detailed: Human: 5/10, CLIP: 30.34
        \end{itemize}
    \item Castle
        \begin{itemize}
            \item Simple: Human: 8/10, CLIP: 29.36
            \item Medium: Human: 7/10, CLIP: 31.98
            \item Detailed: Human: 9/10, CLIP: 30.38
        \end{itemize}
    \item Cyberpunk
        \begin{itemize}
            \item Simple: Human: 8/10, CLIP: 33.70
            \item Medium: Human: 8/10, CLIP: 34.95
            \item Detailed: Human: 7/10, CLIP: 29.07
        \end{itemize}
    \item Cat-Bird
        \begin{itemize}
            \item Simple: Human: 2/10, CLIP: 25.97
            \item Medium: Human: 2/10, CLIP: 27.57
            \item Detailed: Human: 6/10, CLIP: 37.96
        \end{itemize}
\end{itemize}

\begin{figure}[h]
    \centering
    \includegraphics[width=0.3\textwidth]{../results/step3_results/part_b/part_b_space_exploration_simple.png}
    \includegraphics[width=0.3\textwidth]{../results/step3_results/part_b/part_b_space_exploration_medium.png}
    \includegraphics[width=0.3\textwidth]{../results/step3_results/part_b/part_b_space_exploration_detailed.png}
    \caption{Space. \textbf{Left:} Simple - "An astronaut in space" (Human: 5/10, CLIP: 28.49). \textbf{Center:} Medium - "An astronaut in a spacesuit floating above Earth" (Human: 8/10, CLIP: 32.65). \textbf{Right:} Detailed - "A single astronaut in a shiny white spacesuit drifting serenely against the stars in the sky. There is a silent planet below with swirling clouds and blue oceans, with their ship orbiting in the distance" (Human: 9/10, CLIP: 33.75).}
\end{figure}

\begin{figure}[h]
    \centering
    \includegraphics[width=0.3\textwidth]{../results/step3_results/part_b/part_b_underwater_world_simple.png}
    \includegraphics[width=0.3\textwidth]{../results/step3_results/part_b/part_b_underwater_world_medium.png}
    \includegraphics[width=0.3\textwidth]{../results/step3_results/part_b/part_b_underwater_world_detailed.png}
    \caption{Ocean. \textbf{Left:} Simple - "A coral reef" (Human: 9/10, CLIP: 31.39). \textbf{Center:} Medium - "A colorful coral reef with tropical fish of and sunlight filtering through the water" (Human: 6/10, CLIP: 32.93). \textbf{Right:} Detailed - "An underwater coral reef that has all sorts of life, with many fish and sharks swimming around. It has bright corals of all colors and shapes, with sunlight filtering through the clear blue water from above." (Human: 5/10, CLIP: 30.34).}
\end{figure}

\begin{figure}[h]
    \centering
    \includegraphics[width=0.3\textwidth]{../results/step3_results/part_b/part_b_ancient_architecture_simple.png}
    \includegraphics[width=0.3\textwidth]{../results/step3_results/part_b/part_b_ancient_architecture_medium.png}
    \includegraphics[width=0.3\textwidth]{../results/step3_results/part_b/part_b_ancient_architecture_detailed.png}
    \caption{Castle. \textbf{Left:} Simple - "A medieval castle" (Human: 8/10, CLIP: 29.36). \textbf{Center:} Medium - "A lively medieval castle surrounded by a moat and lush greenery" (Human: 7/10, CLIP: 31.98). \textbf{Right:} Detailed - "A beautyful german day, with a large castle made of stone, with a few vines climbing up the spires. The vilage around the castle is full of life, with people walking around the market." (Human: 9/10, CLIP: 30.38).}
\end{figure}

\begin{figure}[h]
    \centering
    \includegraphics[width=0.3\textwidth]{../results/step3_results/part_b/part_b_futuristic_city_simple.png}
    \includegraphics[width=0.3\textwidth]{../results/step3_results/part_b/part_b_futuristic_city_medium.png}
    \includegraphics[width=0.3\textwidth]{../results/step3_results/part_b/part_b_futuristic_city_detailed.png}
    \caption{Cyberpunk. \textbf{Left:} Simple - "A cyberpunk city" (Human: 8/10, CLIP: 33.70). \textbf{Center:} Medium - "A distopian cyberpunk city, with neon lights and flying cars." (Human: 8/10, CLIP: 34.95). \textbf{Right:} Detailed - "A breathtaking cyberpunk megacity that has bustling streets filled with people and vendors. The skyline has many towering skyscrapers, and there are futuristic flying cars." (Human: 7/10, CLIP: 29.07).}
\end{figure}

\begin{figure}[h]
    \centering
    \includegraphics[width=0.3\textwidth]{../results/step3_results/part_b/part_b_fantasy_creatures_simple.png}
    \includegraphics[width=0.3\textwidth]{../results/step3_results/part_b/part_b_fantasy_creatures_medium.png}
    \includegraphics[width=0.3\textwidth]{../results/step3_results/part_b/part_b_fantasy_creatures_detailed.png}
    \caption{Cat-Bird. \textbf{Left:} Simple - "A cat with a bird body" (Human: 2/10, CLIP: 25.97). \textbf{Center:} Medium - "A chimera with the body of a cat, wings of a bird." (Human: 2/10, CLIP: 27.57). \textbf{Right:} Detailed - "A beautiful chimera creature that has the body of a maine coon cat, with large majestic wings of an eagle." (Human: 6/10, CLIP: 37.96).}
\end{figure}


In almost all cases, the most detailed prompts are the best looking images. While the short prompts also did well, the longer prompts alligned better with what I was expecting from the model. I think that in most cases, if you don't know the prompt, the short and long prompts produce images of similar fidelity, which is expected as the model is trained to approach the image manifold similarly, even without a prompt as seen in part A. 


The CLIP and the human scores are pretty well alligned in terms of relitive changes. While the scale of the measurements was not alligned well, when the human score changes, the clip score will usually also have a similar change in score, at least in terms of magnitude. 


% (c) [8 pts] Implement CLIP-based similarity scoring (Radford et al., 2021) to automat-
% ically evaluate how well each generated image aligns with its corresponding prompt.
% Conceptually, CLIP computes embeddings for both the image and the text prompt in
% a shared feature space, and the similarity between them is measured using the cosine
% of the angle between their embeddings.
% Similar to SD, you can find different pre-trained CLIP models (and their corresponding
% model cards with implementation details) in Hugging Face . Compare the CLIP
% similarity scores with your own manually assigned similarity ratings from the previous
% step. Discuss any trends, differences, or insights you observe and reflect on how well
% you believe this metric aligns with human judgment.


\subsection{(D) Negative Prompts}
% (d) [4 pts] Now select one of these 15 prompts and come up with a meaningful negative
% prompt to feed to SD. Fix number of steps to 50 and η = 0.0 but select ωCFG ∈
% {0.0, 2.0, 5.0, 8.0, 12.0, 15.0}. Report the generated images and their corresponding
% CLIP similarity scores. Comment on them.

I chose the prompt: "An astronaut in a spacesuit floating above Earth" with the negative prompt: "blob, blurry, low-definition, water, clouds, fingers" because the original image had a odd blob on the plannet, and I wanted to see if it could remove the fingers. The results are shown in figure \ref{fig:q3D}.

CFG Scale vs CLIP Score results:
\begin{itemize}
    \item CFG 0.0: CLIP Score 32.8\%
    \item CFG 2.0: CLIP Score 33.9\%
    \item CFG 5.0: CLIP Score 33.6\%
    \item CFG 8.0: CLIP Score 35.6\%
    \item CFG 12.0: CLIP Score 33.7\%
    \item CFG 15.0: CLIP Score 33.0\%
\end{itemize}

At CFG 0, the image is very noisy and low quality, and it does not look like it is on the image manifold. As the CFG increases, the astronaut looks more realistic, where at 5 it looks the best. At 5, there is no longer any water on the plannet, and the fingers are gone as was in the negative prompt. At higher CFG, the image becomes a bit degenerate, where the saturation increases, where at 15 the planet is no longer round. This makes sense because at high CFG, there is too much emphasis on the prompt, potentially driving the image off the manifold.

\begin{figure*}[h]
    \centering
    \includegraphics[width=0.3\textwidth]{../results/step3_results/part_d/part_d_cfg_0.0.png}
    \includegraphics[width=0.3\textwidth]{../results/step3_results/part_d/part_d_cfg_2.0.png}
    \includegraphics[width=0.3\textwidth]{../results/step3_results/part_d/part_d_cfg_5.0.png}\\
    \includegraphics[width=0.3\textwidth]{../results/step3_results/part_d/part_d_cfg_8.0.png}
    \includegraphics[width=0.3\textwidth]{../results/step3_results/part_d/part_d_cfg_12.0.png}
    \includegraphics[width=0.3\textwidth]{../results/step3_results/part_d/part_d_cfg_15.0.png}
    \caption{Negative Prompt Results for "An astronaut in a spacesuit floating above Earth" with various CFG scales. From top left to bottom right: CFG 0.0 (CLIP: 32.8\%), CFG 2.0 (CLIP: 33.9\%), CFG 5.0 (CLIP: 33.6\%), CFG 8.0 (CLIP: 35.6\%), CFG 12.0 (CLIP: 33.7\%), CFG 15.0 (CLIP: 33.0\%).}
    \label{fig:q3D}
\end{figure*}