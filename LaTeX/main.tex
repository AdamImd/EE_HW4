\documentclass[journal]{IEEEtran}
% \documentclass[draft]{IEEEtran}
\usepackage{amsmath,amsfonts}
\usepackage{algorithmic}
\usepackage{algorithm}
\usepackage{array}
\usepackage[caption=false,font=normalsize,labelfont=sf,textfont=sf]{subfig}
\usepackage{textcomp}
\usepackage{stfloats}
\usepackage{url}
\usepackage{verbatim}
\usepackage{graphicx}
\usepackage{cite}
\usepackage{float}
% Define todo as a checkbox symbol
\usepackage{amssymb}
\newcommand{\todo}{\(\square\) }
\newcommand{\done}{\(\blacksquare\) }


% or
% \usepackage[draft]{graphicx}


\begin{document}

\title{Conditional Generation for Inverse Problems and Class/Text-Based Conditioning}

\author{Adam Imdieke (imdie022@umn.edu)}

% The paper headers
\markboth{Homework 4}%
{Shell \MakeLowercase{\textit{et al.}}: Bare}

\maketitle


\begin{abstract}
This report contains the implementation and results for several inverse problems and conditional generation methods for modern diffusion models. Section 1 covers: "Posterior Samplers for Inverse Problems", Section 2 covers: "Classifier \& Classifier-Free Guidance", and Section 3 covers: "Text-Based Conditioning with Stable Diffusion". Each section contains implementation details, results, and analysis/discussion of the findings. 
\end{abstract}

\subsection{Posterior Samplers for Inverse Problems}
for each of the following tasks, I got results for three images from each of the two datastes, for each of the four inverse problems. I will only be including one image from each dataset for each task in the report, but the full results are included in the zip file submission. Please refer to the appendix for the images corresponding to each method and dataset. I will report the PSNR, SSIM, and LPIPS for the images shown in the report, but the full results across all three images are also included in the zip file submission.

I used my lab's server to generate the images for this task, which has an NVIDIA A6000, with 1TB ram and 128 CPU cores. 
% Step 1 (35 points): Posterior Samplers for Inverse Problems
% Your first task is to implement and compare different posterior sampling methods for solving
% inverse problems using a pre-trained ADM model1 (Dhariwal & Nichol, 2021) from HW3.
% However, unlike unconditional generation in HW3, posterior sampling aims to reconstruct an
% image that is both consistent with a given measurement (e.g., noisy masked or low resolution
% image) and plausible under the learned data prior.
% To help you get started, we have provided the skeleton code for restoration (hw4 step1 main.py)
% along with the ADM model codes and some helper functions (utils.py) under the folder
% step1 utils. Make sure to check the lines with the #TODO flag inside the hw4 step1 main.py
% file in order to complete the code. Some pointers about the model and code:
% † We have provided the CelebA-HQ and ImageNet pretrained ADM checkpoints (click).
% Please download these .pt files and place them under "./step1 utils/models/".
% 1https://github.com/openai/guided-diffusion
% 1
% † This code builds upon the unconditional generation framework from HW3 and uses the
% same ADM backbone, but trained on a different dataset. In this step, you will extend
% that framework from pure image generation to posterior sampling, where the diffusion
% model is conditioned on measurement consistency. Therefore, most of the information
% from the ADM setup in HW3 remains applicable.
% Algorithm 1 Posterior Sampling Strategies
% Require: T, y, { ˜σi}T
% i=1, η
% 1: xT ∼ N (0, I)
% 2: for t = T, ..., 1 do
% 3: ˆx0|t ← 1√ ¯αt · (xt − ˆϵθ(xt, t)√1 − ¯αt) ▷ Tweedie denoised estimate
% 4: x′
% t−1 ∼ p(xt−1|xt, ˆx0|t) ▷ DDIM sampling
% 5: xt−1 ← Projection or gradient update via x′
% t−1 and ˆx0|t to get closer to {x|y = Ax}
% 6: end for
% 7: return x0
% Figure 2: Four degradations that will be covered in this HW. You can change the box
% indices as long as you keep it 128×128 and at least 10 pixels away from the edges.
% For this step, your goals are:

\subsubsection{A: Predict $\hat{x_0}$ and sample ddim}
% (a) [4 pts] Implement predict x0 hat() and sample ddim() functions, effectively repro-
% ducing Alg. 1 excluding line 5, to ensure you can perform unconditional sampling from
% both pretrained ADM models. Use η = 1.0 throughout this assignment, thereby im-
% plementing the DDPM variant. Run the sampling process for 1000 steps and present
% five different samples from each network in a 2 × 5 grid.
% 2

I implemened the DDIM from HW3, the results are shown in Figure~\ref{fig:uncond_samples}. The images generated look qualitatively similar to those from HW3, and the model for each dataset creates images from that distrobution.

\begin{figure*}[ht]
    \centering
    % CelebA-HQ samples (top row)
    \includegraphics[width=0.18\textwidth]{images/step1_results/CelebA_HQ/unconditional/sample_1.png}\hspace{2mm}
    \includegraphics[width=0.18\textwidth]{images/step1_results/CelebA_HQ/unconditional/sample_2.png}\hspace{2mm}
    \includegraphics[width=0.18\textwidth]{images/step1_results/CelebA_HQ/unconditional/sample_3.png}\hspace{2mm}
    \includegraphics[width=0.18\textwidth]{images/step1_results/CelebA_HQ/unconditional/sample_4.png}\hspace{2mm}
    \includegraphics[width=0.18\textwidth]{images/step1_results/CelebA_HQ/unconditional/sample_5.png}
    \\[2mm]
    % ImageNet samples (bottom row)
    \includegraphics[width=0.18\textwidth]{images/step1_results/ImageNet/unconditional/sample_1.png}\hspace{2mm}
    \includegraphics[width=0.18\textwidth]{images/step1_results/ImageNet/unconditional/sample_2.png}\hspace{2mm}
    \includegraphics[width=0.18\textwidth]{images/step1_results/ImageNet/unconditional/sample_3.png}\hspace{2mm}
    \includegraphics[width=0.18\textwidth]{images/step1_results/ImageNet/unconditional/sample_4.png}\hspace{2mm}
    \includegraphics[width=0.18\textwidth]{images/step1_results/ImageNet/unconditional/sample_5.png}
    \caption{Unconditional samples from CelebA-HQ (top row) and ImageNet (bottom row) pretrained ADM models using DDIM sampling with 1000 steps.}
    \label{fig:uncond_samples}
\end{figure*}




\subsubsection{B: Implement and compare posterior samplers}
% (b) [8 pts] As discussed in the class, earlier attempts focused on direct projections onto
% the constraint set. For example, Iterative Latent Variable Refinement (ILVR) (Choi
% et al., 2021) which is initially proposed for super-resolution (SR) tasks considers the
% following update for line 5 (Alg. 1):
% xt−1 = x′
% t−1 + ζILVR · A†(yt−1 − Ax′
% t−1),
% where ζILVR is the weighting parameter, A† is the left-pseudoinverse of the forward
% operator, and yt−1 ∼ q(yt−1|y0).
% • Select 3 images from each dataset’s validation set provided to you (click) and
% place them in their corresponding folders under "./step1 utils/data/". Note:
% Their pre-trained networks are different so do not forget to change the --dataset
% parser for the correct network prior. Also do not forget to have some fun so choose
% the celebrities you know! Let’s see if you can recognize them after reconstruction!
% • Implement the q sample() function to perform the forward noising process de-
% fined as q(xt−1 | x0) = N (xt−1; √¯αt−1 x0, (1 − ¯αt−1)I).
% • Implement the ilvr() function to perform ILVR’s ∇xt p(y|x) update. Use the
% implemented q sample() function to obtain the noisy measurements yt−1.
% • Perform image restoration for the following inverse problem tasks: (i) SR×4, (ii)
% SR×8, (iii) 80% random inpainting, and (iv) 128×128 box inpainting. Assume
% σy = 0 and use 1000 sampling steps. Tune ζILVR heuristically for best per-
% formance, and report reference, measurement, and reconstruction for each image
% along with their corresponding PSNR, SSIM, and LPIPS metrics. Note:
% These degradations are already defined for you. You only need to change them
% from parser operations.
% • Discuss the observed performance and report the restoration time per image.
% Do you obtain similar results for inpainting tasks as for super-resolution tasks? If
% not, why might that be the case?

I found that an ILVR weight of 0.8 worked well during initial tests, so I used that value for all tasks. For this task I inlcuded the results from CelebA-HQ and ImageNet datasets in Figures~\ref{fig:ilvr_celeba} and \ref{fig:ilvr_imagenet}. 

The CelebA model had an average performance of:
\begin{list}{-}{ }
    \item Time: 50s
    \item SRx4:
    \subitem PSNR: 30.97
    \subitem SSIM: 0.880
    \subitem LPIPS: 0.0729
    \item SRx8:
    \subitem PSNR: 26.44
    \subitem SSIM: 0.7464
    \subitem LPIPS: 0.1243
    \item 80\% random inpainting:
    \subitem PSNR: 20.86
    \subitem SSIM: 0.5541
    \subitem LPIPS: 0.4891
    \item 128x128 box inpainting:
    \subitem PSNR: 20.33
    \subitem SSIM: 0.8244
    \subitem LPIPS: 0.1231
\end{list}

The ImageNet model had an average performance of:
\begin{list}{-}{ }
    \item Time: 176s
    \item SRx4:
        \subitem PSNR: 25.33
        \subitem SSIM: 0.735
        \subitem LPIPS: 0.2653
    \item SRx8:
        \subitem PSNR: 22.10
        \subitem SSIM: 0.5647
        \subitem LPIPS: 0.2642
    \item 80\% random inpainting:
        \subitem PSNR: 15.95
        \subitem SSIM: 0.1889
        \subitem LPIPS: 0.9146
    \item 128x128 box inpainting:
        \subitem PSNR: 17.23
        \subitem SSIM: 0.7826
        \subitem LPIPS: 0.2218
\end{list}


\subsubsection{C: Manifold Constrained Gradient (MCG)}  
% (c) [6 pts] Manifold Constrained Gradient (MCG) (Chung et al., 2022) improves upon
% ILVR by first taking a gradient step along the manifold followed by a projection step
% similar to ILVR (but with A⊤ instead of A†):
% ˜xt−1 = x′
% t−1 − ζMCG · ∇xt ||y − Aˆx0|t||2
% xt−1 = x′
% t−1 + A⊤(yt−1 − A˜xt−1)
% • Implement the mcg() function and perform image restoration for the following
% inverse problem tasks: (i) SR×4, (ii) SR×8, (iii) 80% random inpainting, and
% (iv) 128×128 box inpainting. Assume σy = 0 and use 1000 sampling steps.
% Tune ζMCG heuristically for best performance, and report reference, measurement,
% and reconstruction for each image along with their corresponding PSNR, SSIM,
% and LPIPS metrics. Note: Use the same 3 images from part (b) and obtain
% yt−1 similar to ILVR using q sample() function.
% 3
% • Discuss the observed performance and report the restoration time per image.
% Compare the speed and quality to ILVR. Do you see consistent performance across
% restoration tasks this time? Although MCG also requires 1000 sampling (denois-
% ing) steps, do you notice any slowness compared to ILVR? Discuss.

MCG produced 

\begin{list}{-}{ }
    \item Time: 55s
    \item SRx4:
        \subitem PSNR: 19.68
        \subitem SSIM: 0.8047
        \subitem LPIPS: 0.1846
    \item SRx8:
        \subitem PSNR: 25.68
        \subitem SSIM: 0.7404
        \subitem LPIPS: 0.1335
    \item 80\% random inpainting:
        \subitem PSNR: 33.40
        \subitem SSIM: 0.9260
        \subitem LPIPS: 0.0359
    \item 128x128 box inpainting:
        \subitem PSNR: 19.68
        \subitem SSIM:0.8047
        \subitem LPIPS: 0.1856
\end{list}



\begin{list}{-}{ }
    \item Time: 178s
    \item SRx4:
        \subitem PSNR: 12.75
        \subitem SSIM: 0.3567
        \subitem LPIPS: 0.6224
    \item SRx8:
        \subitem PSNR: 14.46
        \subitem SSIM: 0.3693
        \subitem LPIPS: 0.4716
    \item 80\% random inpainting:
        \subitem PSNR: 23.98
        \subitem SSIM: 0.7838
        \subitem LPIPS: 0.2242
    \item 128x128 box inpainting:
        \subitem PSNR: 14.11
        \subitem SSIM: 0.7472
        \subitem LPIPS: 0.3034
\end{list}




\subsubsection{D: Denoising Diffusion Null-Space Model (DDNM)}
% (d) [6 pts] Denoising Diffusion Null-Space Model (DDNM) (Wang et al., 2023) replaces
% the line 5 update in Alg. 1 with an update from a range-null-space decomposition of
% the forward operator. Specifically, at each step, it forms the Tweedie estimate ˆx0|t and
% projects it onto the affine constraint set as:
% ˜x0|t = ˆx0|t + ζDDNM · A†(y − Aˆx0|t).
% The goal here is to enforce exact data consistency in the range of A, while preserving
% the null-space component favored by the prior. Once this refined denoised estimate
% is obtained, DDNM gets the final sample as x′
% t−1 ∼ p(xt−1|xt, ˜x0|t). Given this new
% formulation, DDNM improves upon past methods in terms of required sampling steps.
% • Implement the ddnm() function and perform image restoration for the following
% inverse problem tasks: (i) SR×4, (ii) SR×8, (iii) 80% random inpainting, and (iv)
% 128×128 box inpainting. Assume σy = 0 and use 100 sampling steps. Tune
% ζDDNM heuristically for best performance, and report reference, measurement, and
% reconstruction for each image along with their corresponding PSNR, SSIM, and
% LPIPS metrics. Note: Use the same 3 images from part (b).
% • Discuss the observed performance and report the restoration time per image.
% Compare the speed and quality to ILVR and MCG.

\begin{list}{-}{ }
    \item Time: 3s
    \item SRx4:
        \subitem PSNR: 18.25
        \subitem SSIM: 0.2746
        \subitem LPIPS: 0.8742
    \item SRx8:
        \subitem PSNR: 17.02
        \subitem SSIM: 0.2033
        \subitem LPIPS: 1.0188
    \item 80\% random inpainting:
        \subitem PSNR: 10.87
        \subitem SSIM: 0.0629
        \subitem LPIPS: 14.827
    \item 128x128 box inpainting:
        \subitem PSNR: 15.83
        \subitem SSIM: 0.7219
        \subitem LPIPS: 0.5116
\end{list}

\begin{list}{-}{ }
    \item Time: 9s
    \item SRx4:
        \subitem PSNR: 13.33
        \subitem SSIM: 0.3296
        \subitem LPIPS: 0.8221
    \item SRx8:
        \subitem PSNR: 12.76
        \subitem SSIM: 0.2855
        \subitem LPIPS: 0.9485
    \item 80\% random inpainting:
        \subitem PSNR: 8.03
        \subitem SSIM: 0.0324
        \subitem LPIPS: 1.4114
    \item 128x128 box inpainting:
        \subitem PSNR: 13.24
        \subitem SSIM: 0.7202
        \subitem LPIPS: 0.5390
\end{list}

\subsubsection{E: Noisy measurements and Diffusion Posterior Sampling (DPS)}
% (e) [5 pts] Add noise to the measurements with σy = 0.05 (use --sigma y). Repeat steps
% (b), (c), and (d) only for the SR×4 and box inpainting tasks on a single dataset (either
% CelebA-HQ or ImageNet). Report reference, measurement, and reconstruction for each
% image. Do you observe any performance degradation for ILVR, MCG, or DDNM? Is
% the drop in performance more pronounced for one inverse task compared to the other
% (SR×4 vs box inpainting)? Explain your reasoning for each case.

% \begin{figure}
%     \centering
%     \includegraphics[width=0.9\columnwidth]{images/step1_results/CelebA_HQ_1E/Noisy_ILVR/SR_x4/recon_3.png}
%     \includegraphics[width=0.9\columnwidth]{images/step1_results/CelebA_HQ/Noisy_ILVR/Inpainting_box/recon_3.png}
%     \caption{ILVR Reconstructions on CelebA-HQ with noisy measurements for SRx4 and 128x128 box inpainting.}
%     \label{fig:noisy_ilvr_celeba}
% \end{figure}


\begin{list}{-}{ }
    \item Time:
    \item SRx4:
        \subitem PSNR:
        \subitem SSIM:
        \subitem LPIPS:
    \item SRx8:
        \subitem PSNR:
        \subitem SSIM:
        \subitem LPIPS:
    \item 80\% random inpainting:
        \subitem PSNR:
        \subitem SSIM:
        \subitem LPIPS:
    \item 128x128 box inpainting:
        \subitem PSNR:
        \subitem SSIM:
        \subitem LPIPS:
\end{list}


\begin{list}{-}{ }
    \item Time:
    \item SRx4:
        \subitem PSNR:
        \subitem SSIM:
        \subitem LPIPS:
    \item SRx8:
        \subitem PSNR:
        \subitem SSIM:
        \subitem LPIPS:
    \item 80\% random inpainting:
        \subitem PSNR:
        \subitem SSIM:
        \subitem LPIPS:
    \item 128x128 box inpainting:
        \subitem PSNR:
        \subitem SSIM:
        \subitem LPIPS:
\end{list}


\subsubsection{F: Diffusion Posterior Sampling (DPS) for noisy measurements}
% (f) [6 pts] Diffusion Posterior Sampling (DPS) (Chung et al., 2023) applies MCG up-
% date for the most part but it removes the final projection step for stability to noisy
% conditions. Therefore, its update is given as:
% xt−1 = x′
% t−1 − ζDPS · ∇xt ||y − Aˆx0|t||2
% • Implement the dps() function and perform image restoration for the following
% inverse problem tasks: (i) SR×4, (ii) SR×8, (iii) 80% random inpainting, and
% (iv) 128×128 box inpainting. Assume σy = 0.05 and use 1000 sampling steps.
% Tune ζDPS heuristically for best performance, and report reference, measurement,
% and reconstruction for each image along with their corresponding PSNR, SSIM,
% and LPIPS metrics. Note: Use the same 3 images from part (b).
% • Discuss the observed performance and report the restoration time per image.
% Compare the speed and quality to ILVR, MCG and DDNM. Do you see improved
% performance across noisy conditions? Discuss


\begin{list}{-}{ }
    \item Time: 55s
    \item SRx4:
        \subitem PSNR: 22.94
        \subitem SSIM: 0.7031
        \subitem LPIPS: 0.2730
    \item SRx8:
        \subitem PSNR: 26.43
        \subitem SSIM: 0.7592
        \subitem LPIPS: 0.1143
    \item 80\% random inpainting:
        \subitem PSNR: 19.02
        \subitem SSIM: 0.7389
        \subitem LPIPS: 0.3173
    \item 128x128 box inpainting:
        \subitem PSNR: 11.92
        \subitem SSIM: 0.3863
        \subitem LPIPS: 0.5397
\end{list}


\begin{list}{-}{ }
    \item Time: 177s
    \item SRx4:
        \subitem PSNR: 13.22
        \subitem SSIM: 0.3813
        \subitem LPIPS: 0.4515
    \item SRx8:
        \subitem PSNR: 16.65
        \subitem SSIM: 0.4041
        \subitem LPIPS: 0.3862
    \item 80\% random inpainting:
        \subitem PSNR: 19.34
        \subitem SSIM: 0.5619
        \subitem LPIPS: 0.4727
    \item 128x128 box inpainting:
        \subitem PSNR: 13.69
        \subitem SSIM: 0.5530
        \subitem LPIPS: 0.4243
\end{list}
\section{Conditioning Diffusion}
% Your next task is to explore two key conditioning mechanisms used in modern diffusion
% models: classifier guidance (CG) (Dhariwal & Nichol, 2021) and classifier-free guidance
% (CFG) (Ho & Salimans, 2021). Both techniques allow diffusion models to steer the generation
% process toward a desired class or concept, but they achieve this through fundamentally
% different principles.
% To help you get started, we have provided the skeleton code for class-conditional sampling
% (hw4 step2 main.py) along with a lightweight diffusion model and a noise-aware classifier
% codes under the folder step2 utils. We further provided pre-trained networks for each but
% you can also train them from scratch if desired.
% Your task is to complete the implementation of the two missing sampling methods. Make
% sure to check the lines with the #TODO flag inside the hw4 step2 main.py file in order to
% complete the code. Your goals for this step are as follows:



% (a) [7 pts] Implement sample cg() to perform Classifier Guidance (CG) sampling, where
% the guidance signal is obtained from the gradient of the noise-aware classifier:
% √
% ϵ̃ = ϵθ (xt , t, ∅) − ωCG · 1 − ᾱt ∇xt log pϕ (y|xt , t),
% where ωCG is the guidance scale (provided as --cg scale in the parser). Use the reverse
% update rule of the DDPM parameterization to generate class-conditioned samples for
% digits 0–9. Present the generated digits in a 10 × 10 grid (each row represents 10
% samples from a single digit), varying ωCG ∈ {0.0, 1.0, 3.0, 5.0, 10.0} to visualize the
% strength of guidance.
% (b) [7 pts] Implement sample cfg() to perform Classifier-Free Guidance (CFG) sampling.
% Use the unconditional and conditional noise predictions from the diffusion model and
% combine them according to:
% ϵ̃ = ϵθ (xt , t, ∅) + ωCFG · (ϵθ (xt , t, c) − ϵθ (xt , t, ∅)),
% where ωCFG is the guidance scale (provided as --cfg scale in the parser). Follow the
% same reverse update as in the DDPM case, and visualize the generated digits in a
% 10 × 10 grid for ωCG ∈ {0.0, 1.0, 3.0, 5.0, 10.0}.
% (c) [6 pts] Compare the outputs of CFG and CG both qualitatively and quantitatively.
% For each method, compute:
% • Classification accuracy of the generated images using the provided noise-aware
% classifier.
% • Intra-class diversity using the intraclass diversity cosine() metric (already
% implemented for you).
% Reflect on your findings. How does the guidance scale influence sample quality and
% diversity in each method? Which method do you find more stable or visually consistent
% across classes? Based on your implementations and observations, explain the funda-
% mental difference between CG and CFG. Why do you think CFG is more popular for
% text-to-image generation?
\section{Text to image Generation}

\subsection{(A)No Prompt}
% (a) [2 pts] Run your pipeline using ωCFG = 0.0, η = 0.0 and 50 sampling steps without
% any positive or negative prompt to make sure everything works (this should give you
% a meaningful image generated unconditionally).

For the no-prompt generation, I used the same model as in part B, with the parameters $\omega CFG = 0.0$, num steps = 50, and $eta = 0.0$, with a fixed seed at 42. The resulting image is shown below.

\begin{figure}[h]
    \centering
    \includegraphics[width=0.4\textwidth]{../results/step3_results/part_a/part_a_unconditional.png}
    \caption{No Prompt Image Generation}
\end{figure}



\subsection{(B-C) 5X3 prompts- Manual evaluation}

% (b) [6 pts] Create 15 engaging prompts based on 5 unique and different topics (see some
% examples in (b) figure). For each topic, write three versions of the same idea:
% 1. Simple and short: a short, clear prompt (about one sentence).
% 2. Medium: a slightly longer prompt that adds some detail or context.
% 3. Long and complex: a detailed, imaginative version that expands on the idea
% with creative or descriptive elements.
% 2
% https://huggingface.co/docs/diffusers
% 6You will end up with 5 topics × 3 versions = 15 prompts in total. Do not rely on
% LLMs to generate you these prompts and use your creativity to generate interesting
% prompts that you want to see visualized as images. For each prompt, generate the
% corresponding SD output. Use ωCFG = 10.0 with 50 sampling steps and η = 0.0. Keep
% the negative prompt NULL. Finally, assign each generated image a similarity score (out
% of 10) based on how well it matches the intended prompt (e.g., 5.4/10.0). Share your
% thoughts. For instance, which version of a prompt typically gave the best similarity
% or image quality? Does prompt length affect fidelity to the intended concept or the
% final image quality?
I implemented a pipeline that will read from a set of prompts, and then use them to generate the images from the diffusers StableDiffusionPipeline module, with the $\omega CFG = 10$, num steps = 50, $eta = 0.0$, and a fixed seed at 42. The prompts for each image are included in the caption of the figures. 

\begin{itemize}
    \item Space
        \begin{itemize}
            \item Simple: Human: 5/10, CLIP: 28.49
            \item Medium: Human: 8/10, CLIP: 32.65
            \item Detailed: Human: 9/10, CLIP: 33.75
        \end{itemize}
    \item Ocean
        \begin{itemize}
            \item Simple: Human: 9/10, CLIP: 31.39
            \item Medium: Human: 6/10, CLIP: 32.93
            \item Detailed: Human: 5/10, CLIP: 30.34
        \end{itemize}
    \item Castle
        \begin{itemize}
            \item Simple: Human: 8/10, CLIP: 29.36
            \item Medium: Human: 7/10, CLIP: 31.98
            \item Detailed: Human: 9/10, CLIP: 30.38
        \end{itemize}
    \item Cyberpunk
        \begin{itemize}
            \item Simple: Human: 8/10, CLIP: 33.70
            \item Medium: Human: 8/10, CLIP: 34.95
            \item Detailed: Human: 7/10, CLIP: 29.07
        \end{itemize}
    \item Cat-Bird
        \begin{itemize}
            \item Simple: Human: 2/10, CLIP: 25.97
            \item Medium: Human: 2/10, CLIP: 27.57
            \item Detailed: Human: 6/10, CLIP: 37.96
        \end{itemize}
\end{itemize}

\begin{figure}[h]
    \centering
    \includegraphics[width=0.3\textwidth]{../results/step3_results/part_b/part_b_space_exploration_simple.png}
    \includegraphics[width=0.3\textwidth]{../results/step3_results/part_b/part_b_space_exploration_medium.png}
    \includegraphics[width=0.3\textwidth]{../results/step3_results/part_b/part_b_space_exploration_detailed.png}
    \caption{Space. \textbf{Left:} Simple - "An astronaut in space" (Human: 5/10, CLIP: 28.49). \textbf{Center:} Medium - "An astronaut in a spacesuit floating above Earth" (Human: 8/10, CLIP: 32.65). \textbf{Right:} Detailed - "A single astronaut in a shiny white spacesuit drifting serenely against the stars in the sky. There is a silent planet below with swirling clouds and blue oceans, with their ship orbiting in the distance" (Human: 9/10, CLIP: 33.75).}
\end{figure}

\begin{figure}[h]
    \centering
    \includegraphics[width=0.3\textwidth]{../results/step3_results/part_b/part_b_underwater_world_simple.png}
    \includegraphics[width=0.3\textwidth]{../results/step3_results/part_b/part_b_underwater_world_medium.png}
    \includegraphics[width=0.3\textwidth]{../results/step3_results/part_b/part_b_underwater_world_detailed.png}
    \caption{Ocean. \textbf{Left:} Simple - "A coral reef" (Human: 9/10, CLIP: 31.39). \textbf{Center:} Medium - "A colorful coral reef with tropical fish of and sunlight filtering through the water" (Human: 6/10, CLIP: 32.93). \textbf{Right:} Detailed - "An underwater coral reef that has all sorts of life, with many fish and sharks swimming around. It has bright corals of all colors and shapes, with sunlight filtering through the clear blue water from above." (Human: 5/10, CLIP: 30.34).}
\end{figure}

\begin{figure}[h]
    \centering
    \includegraphics[width=0.3\textwidth]{../results/step3_results/part_b/part_b_ancient_architecture_simple.png}
    \includegraphics[width=0.3\textwidth]{../results/step3_results/part_b/part_b_ancient_architecture_medium.png}
    \includegraphics[width=0.3\textwidth]{../results/step3_results/part_b/part_b_ancient_architecture_detailed.png}
    \caption{Castle. \textbf{Left:} Simple - "A medieval castle" (Human: 8/10, CLIP: 29.36). \textbf{Center:} Medium - "A lively medieval castle surrounded by a moat and lush greenery" (Human: 7/10, CLIP: 31.98). \textbf{Right:} Detailed - "A beautyful german day, with a large castle made of stone, with a few vines climbing up the spires. The vilage around the castle is full of life, with people walking around the market." (Human: 9/10, CLIP: 30.38).}
\end{figure}

\begin{figure}[h]
    \centering
    \includegraphics[width=0.3\textwidth]{../results/step3_results/part_b/part_b_futuristic_city_simple.png}
    \includegraphics[width=0.3\textwidth]{../results/step3_results/part_b/part_b_futuristic_city_medium.png}
    \includegraphics[width=0.3\textwidth]{../results/step3_results/part_b/part_b_futuristic_city_detailed.png}
    \caption{Cyberpunk. \textbf{Left:} Simple - "A cyberpunk city" (Human: 8/10, CLIP: 33.70). \textbf{Center:} Medium - "A distopian cyberpunk city, with neon lights and flying cars." (Human: 8/10, CLIP: 34.95). \textbf{Right:} Detailed - "A breathtaking cyberpunk megacity that has bustling streets filled with people and vendors. The skyline has many towering skyscrapers, and there are futuristic flying cars." (Human: 7/10, CLIP: 29.07).}
\end{figure}

\begin{figure}[h]
    \centering
    \includegraphics[width=0.3\textwidth]{../results/step3_results/part_b/part_b_fantasy_creatures_simple.png}
    \includegraphics[width=0.3\textwidth]{../results/step3_results/part_b/part_b_fantasy_creatures_medium.png}
    \includegraphics[width=0.3\textwidth]{../results/step3_results/part_b/part_b_fantasy_creatures_detailed.png}
    \caption{Cat-Bird. \textbf{Left:} Simple - "A cat with a bird body" (Human: 2/10, CLIP: 25.97). \textbf{Center:} Medium - "A chimera with the body of a cat, wings of a bird." (Human: 2/10, CLIP: 27.57). \textbf{Right:} Detailed - "A beautiful chimera creature that has the body of a maine coon cat, with large majestic wings of an eagle." (Human: 6/10, CLIP: 37.96).}
\end{figure}


In almost all cases, the most detailed prompts are the best looking images. While the short prompts also did well, the longer prompts alligned better with what I was expecting from the model. I think that in most cases, if you don't know the prompt, the short and long prompts produce images of similar fidelity, which is expected as the model is trained to approach the image manifold similarly, even without a prompt as seen in part A. 


The CLIP and the human scores are pretty well alligned in terms of relitive changes. While the scale of the measurements was not alligned well, when the human score changes, the clip score will usually also have a similar change in score, at least in terms of magnitude. 


% (c) [8 pts] Implement CLIP-based similarity scoring (Radford et al., 2021) to automat-
% ically evaluate how well each generated image aligns with its corresponding prompt.
% Conceptually, CLIP computes embeddings for both the image and the text prompt in
% a shared feature space, and the similarity between them is measured using the cosine
% of the angle between their embeddings.
% Similar to SD, you can find different pre-trained CLIP models (and their corresponding
% model cards with implementation details) in Hugging Face . Compare the CLIP
% similarity scores with your own manually assigned similarity ratings from the previous
% step. Discuss any trends, differences, or insights you observe and reflect on how well
% you believe this metric aligns with human judgment.


\subsection{(D) Negative Prompts}
% (d) [4 pts] Now select one of these 15 prompts and come up with a meaningful negative
% prompt to feed to SD. Fix number of steps to 50 and η = 0.0 but select ωCFG ∈
% {0.0, 2.0, 5.0, 8.0, 12.0, 15.0}. Report the generated images and their corresponding
% CLIP similarity scores. Comment on them.

I chose the prompt: "An astronaut in a spacesuit floating above Earth" with the negative prompt: "blob, blurry, low-definition, water, clouds, fingers" because the original image had a odd blob on the plannet, and I wanted to see if it could remove the fingers. The results are shown in figure \ref{fig:q3D}.

CFG Scale vs CLIP Score results:
\begin{itemize}
    \item CFG 0.0: CLIP Score 32.8\%
    \item CFG 2.0: CLIP Score 33.9\%
    \item CFG 5.0: CLIP Score 33.6\%
    \item CFG 8.0: CLIP Score 35.6\%
    \item CFG 12.0: CLIP Score 33.7\%
    \item CFG 15.0: CLIP Score 33.0\%
\end{itemize}

At CFG 0, the image is very noisy and low quality, and it does not look like it is on the image manifold. As the CFG increases, the astronaut looks more realistic, where at 5 it looks the best. At 5, there is no longer any water on the plannet, and the fingers are gone as was in the negative prompt. At higher CFG, the image becomes a bit degenerate, where the saturation increases, where at 15 the planet is no longer round. This makes sense because at high CFG, there is too much emphasis on the prompt, potentially driving the image off the manifold.

\begin{figure*}[h]
    \centering
    \includegraphics[width=0.3\textwidth]{../results/step3_results/part_d/part_d_cfg_0.0.png}
    \includegraphics[width=0.3\textwidth]{../results/step3_results/part_d/part_d_cfg_2.0.png}
    \includegraphics[width=0.3\textwidth]{../results/step3_results/part_d/part_d_cfg_5.0.png}\\
    \includegraphics[width=0.3\textwidth]{../results/step3_results/part_d/part_d_cfg_8.0.png}
    \includegraphics[width=0.3\textwidth]{../results/step3_results/part_d/part_d_cfg_12.0.png}
    \includegraphics[width=0.3\textwidth]{../results/step3_results/part_d/part_d_cfg_15.0.png}
    \caption{Negative Prompt Results for "An astronaut in a spacesuit floating above Earth" with various CFG scales. From top left to bottom right: CFG 0.0 (CLIP: 32.8\%), CFG 2.0 (CLIP: 33.9\%), CFG 5.0 (CLIP: 33.6\%), CFG 8.0 (CLIP: 35.6\%), CFG 12.0 (CLIP: 33.7\%), CFG 15.0 (CLIP: 33.0\%).}
    \label{fig:q3D}
\end{figure*}

\begin{figure}[h]
    \centering
    \includegraphics[width=0.9\columnwidth]{images/step1_results/CelebA_HQ/ILVR/SR_x4/recon_3.png}
    \includegraphics[width=0.9\columnwidth]{images/step1_results/CelebA_HQ/ILVR/SR_x8/recon_3.png}
    \includegraphics[width=0.9\columnwidth]{images/step1_results/CelebA_HQ/ILVR/Inpainting_random_80pct/recon_3.png}
    \includegraphics[width=0.9\columnwidth]{images/step1_results/CelebA_HQ/ILVR/Inpainting_box/recon_3.png}
    \caption{ILVR Reconstructions on CelebA-HQ for SRx4, SRx8, 80\% random inpainting, and 128x128 box inpainting.}
    \label{fig:ilvr_celeba}
\end{figure}


\begin{figure}[h]
    \centering
    \includegraphics[width=0.9\columnwidth]{images/step1_results/ImageNet/ILVR/SR_x4/recon_1.png}
    \includegraphics[width=0.9\columnwidth]{images/step1_results/ImageNet/ILVR/SR_x8/recon_1.png}
    \includegraphics[width=0.9\columnwidth]{images/step1_results/ImageNet/ILVR/Inpainting_random_80pct/recon_1.png}
    \includegraphics[width=0.9\columnwidth]{images/step1_results/ImageNet/ILVR/Inpainting_box/recon_1.png}
    \caption{ILVR Reconstructions on ImageNet for SRx4, SRx8, 80\% random inpainting, and 128x128 box inpainting.}
    \label{fig:ilvr_imagenet}
\end{figure}








\begin{figure}
    \centering
    \includegraphics[width=0.9\columnwidth]{images/step1_results/CelebA_HQ/MCG/Inpainting_box/recon_3.png}
    \includegraphics[width=0.9\columnwidth]{images/step1_results/CelebA_HQ/MCG/SR_x8/recon_3.png}
    \includegraphics[width=0.9\columnwidth]{images/step1_results/CelebA_HQ/MCG/Inpainting_random_80pct/recon_3.png}
    \includegraphics[width=0.9\columnwidth]{images/step1_results/CelebA_HQ/MCG/Inpainting_box/recon_3.png}
    \caption{MCG Reconstructions on CelebA-HQ for SRx4, SRx8, 80\% random inpainting, and 128x128 box inpainting.}
    \label{fig:mcg_celeba}
\end{figure}

\begin{figure}
    \centering
    \includegraphics[width=0.9\columnwidth]{images/step1_results/ImageNet/MCG/SR_x4/recon_1.png}
    \includegraphics[width=0.9\columnwidth]{images/step1_results/ImageNet/MCG/SR_x8/recon_1.png}
    \includegraphics[width=0.9\columnwidth]{images/step1_results/ImageNet/MCG/Inpainting_random_80pct/recon_1.png}
    \includegraphics[width=0.9\columnwidth]{images/step1_results/ImageNet/MCG/Inpainting_box/recon_1.png}
    \caption{MCG Reconstructions on ImageNet for SRx4, SRx8, 80\% random inpainting, and 128x128 box inpainting.}
    \label{fig:mcg_imagenet}
\end{figure}



\begin{figure}
    \centering
    \includegraphics[width=0.9\columnwidth]{images/step1_results/CelebA_HQ/DDNM/SR_x4/recon_3.png}
    \includegraphics[width=0.9\columnwidth]{images/step1_results/CelebA_HQ/DDNM/SR_x8/recon_3.png}
    \includegraphics[width=0.9\columnwidth]{images/step1_results/CelebA_HQ/DDNM/Inpainting_random_80pct/recon_3.png}
    \includegraphics[width=0.9\columnwidth]{images/step1_results/CelebA_HQ/DDNM/Inpainting_box/recon_3.png}
    \caption{DDNM Reconstructions on CelebA-HQ for SRx4, SRx8, 80\% random inpainting, and 128x128 box inpainting.}
    \label{fig:ddnm_celeba}
\end{figure}

\begin{figure}
    \centering
    \includegraphics[width=0.9\columnwidth]{images/step1_results/ImageNet/DDNM/SR_x4/recon_1.png}
    \includegraphics[width=0.9\columnwidth]{images/step1_results/ImageNet/DDNM/SR_x8/recon_1.png}
    \includegraphics[width=0.9\columnwidth]{images/step1_results/ImageNet/DDNM/Inpainting_random_80pct/recon_1.png}
    \includegraphics[width=0.9\columnwidth]{images/step1_results/ImageNet/DDNM/Inpainting_box/recon_1.png}
    \caption{DDNM Reconstructions on ImageNet for SRx4, SRx8, 80\% random inpainting, and 128x128 box inpainting.}
    \label{fig:ddnm_imagenet}
\end{figure}






\begin{figure}
    \centering
    \includegraphics[width=0.9\columnwidth]{images/step1_results/CelebA_HQ/DPS/SR_x4/recon_3.png}
    \includegraphics[width=0.9\columnwidth]{images/step1_results/CelebA_HQ/DPS/SR_x8/recon_3.png}
    \includegraphics[width=0.9\columnwidth]{images/step1_results/CelebA_HQ/DPS/Inpainting_random_80pct/recon_3.png}
    \includegraphics[width=0.9\columnwidth]{images/step1_results/CelebA_HQ/DPS/Inpainting_box/recon_3.png}
    \caption{DPS Reconstructions on CelebA-HQ for SRx4, SRx8, 80\% random inpainting, and 128x128 box inpainting with noisy measurements.}
    \label{fig:dps_celeba}
\end{figure}

\begin{figure}
    \centering
    \includegraphics[width=0.9\columnwidth]{images/step1_results/ImageNet/DPS/SR_x4/recon_1.png}
    \includegraphics[width=0.9\columnwidth]{images/step1_results/ImageNet/DPS/SR_x8/recon_1.png}
    \includegraphics[width=0.9\columnwidth]{images/step1_results/ImageNet/DPS/Inpainting_random_80pct/recon_1.png}
    \includegraphics[width=0.9\columnwidth]{images/step1_results/ImageNet/DPS/Inpainting_box/recon_1.png}
    \caption{DPS Reconstructions on ImageNet for SRx4, SRx8, 80\% random inpainting, and 128x128 box inpainting with noisy measurements.}
    \label{fig:dps_imagenet}
\end{figure}






\vfill

\end{document}


